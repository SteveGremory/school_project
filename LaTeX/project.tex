\documentclass{article}

\usepackage{minted}
% (2) specify encoding
\usepackage[T1]{fontenc}
% (3) load symbol definitions
\usepackage{textcomp}

\begin{document}


\begin{center}
\huge{Information Pracitices}\\[0.5cm]
\normalsize
\textsc{Shaurya International School}\\[2.0cm]

\emph{\LARGE Certificate}\\[2.5cm]
\end{center}
\normalsize This is to certify that this is a bonafide record of the project presented by Tejveer Singh of class 12th in partial fulfillment of the project requirements.\\[1.0cm]

\vfill

% Bottom of the page
\begin{flushright}
Tejveer Singh \\
Roll Number: 46\\
Class 12th\\
\end{flushright}

\newpage

\begin{abstract}
This is a simple tensorflow model that takes in human written
digits as 28x28 pixel images and tries to guess what the digit is.
The model is very basic and has only been tested on digits 0-9 written
with black color on a white background. Prerequisites: TensorFlow, Numpy,
Matplotlib and a dataset of hand drawn images placed in a folder called
``testdata'' with a csv file called ``testdata.csv'' containing the description
of the data.
\end{abstract}

\section{How does it work?}
The model is first trained on the MNIST dataset which is then normalise
and is sent to a ML model which first passes it through the Flatten layer
and then Dense layer which consists of two RELU (Rectify leniar unit) layers
and a softmax layer which gets the output. Then the model is compiled
using Sparse Categorical Crossentropy as the loss function and Adam is
the optimiser with the accuracy metric. After all these params are
set, the model is trained and saved to the folder ``writtendigits.model.'' \\

After the model has been trained, it is loaded into the program by the
load model function and the data alongwith the correct answers is
also loaded into memory using pandas. Now, the program looks for PNG files
in a folder called ``testdata'' which are named by a certain name convention. \\

These images are now read by OpenCV and inverted by numpy such that instead of
black on white, it's white on black (hence why the model only works with
monochrome images). Now, array containing the image data is given to the model
for it to predict the answer. If the predection is correct, a value of `1' is
pushed to an array declared earlier in the code to keep track of the correct/wrong
answers for later use and a print statement stating that the original number and
the predection is executed. A zero is pushed back to the array if the value is wrong
however the print statement is executed nonetheless.

Lastly, a graph of 0s and 1s is displayed where a peak (1) suggests that the model
predected the answer correctly and a 0 suggests the opposite.

\section{Code}
main.py:
\begin{minted}{python}
from train_model import train_model
import matplotlib.pyplot as plt
import tensorflow as tf
import numpy as np
import pandas
import cv2
import os

# Create a new model if it doesn't exist
if os.path.exists("writtendigits.model"):
    model = tf.keras.models.load_model("writtendigits.model")
else:
    train_model()
    model = tf.keras.models.load_model("writtendigits.model")

# Read the data with the correct information
data = pandas.read_csv("testdata/numbers.csv")
correct = []

for image_number in data["num"]:
    if os.path.exists(f"./testdata/number_{image_number}.png"):

        image = cv2.imread(f"./testdata/number_{image_number}.png")[:, :, 0]
        image = np.invert(np.array([image]))

        predection = model.predict(image)

        if np.argmax(predection) == image_number:
            correct += [1]
        else:
            correct += [0]

        print("The predection is incorrect:")
        print(f"Number: {image_number}\t Predection: {np.argmax(predection)}")

# Display a graph of correct/incorrect numbers
# data to be plotted
x = np.arange(0, 9)
y = np.array(correct)

# plotting
plt.title("Model accuracy")
plt.xlabel("Predection")
plt.ylabel("Number")
plt.plot(x, y, color="green")

plt.show()
\end{minted}

\newpage
\begin{flushleft}
train\_model.py:
\end{flushleft}
\begin{minted}{python}
import tensorflow as tf

def train_model():
    # Get the MNIST dataset
    mnist = tf.keras.datasets.mnist

    # Get the training and testing data
    (x_train, y_train), (x_test, y_test) = mnist.load_data()

    # Normalize the data
    x_train = tf.keras.utils.normalize(x_train, axis=1)
    x_test = tf.keras.utils.normalize(x_test, axis=1)

    # Get a sequential tensorflow model
    model = tf.keras.models.Sequential()
    # Flatten the input shape
    model.add(tf.keras.layers.Flatten(input_shape=(28, 28)))
    # Add the dense layer
    # Rectify leniar unit -> RELU
    model.add(tf.keras.layers.Dense(128, activation="relu"))
    model.add(tf.keras.layers.Dense(128, activation="relu"))
    # Output layer
    # This layer makes sure that all 10 neurons add up to 1
    # Probability for each digit to be the right answer
    model.add(tf.keras.layers.Dense(10, activation="softmax"))
    loss_fn = tf.keras.losses.SparseCategoricalCrossentropy()
    model.compile(optimizer="adam", loss=loss_fn, metrics=["accuracy"])

    # Train the model
    # EPOCHs -> How many times will the model see the same data
    model.fit(x_train, y_train, epochs=3)
    model.save("writtendigits.model")
\end{minted}

\end{document}